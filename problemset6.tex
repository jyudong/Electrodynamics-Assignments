\documentclass[10.5pt]{article}
\usepackage{amsmath,amssymb,amsthm}
\usepackage{listings}
\usepackage{graphicx}
\usepackage[shortlabels]{enumitem}
\usepackage{tikz}
\usepackage[margin=1in]{geometry}
\usepackage{fancyhdr}
\usepackage{epsfig} %% for loading postscript figures
\usepackage{amsmath}
\usepackage{float}
\usepackage{amssymb}
\usepackage{caption}
\usepackage{subfigure}
\usepackage{graphics}
\usepackage{titlesec}
\usepackage{mathrsfs}
\usepackage{amsfonts}
\usepackage{indentfirst}
\usepackage{fancybox}
\usepackage{tikz}
\usepackage{algorithm}
\usepackage{algcompatible}
\usepackage{xeCJK}
\usepackage{extarrows}
\setCJKmainfont{Kai}

\title{Electrodynamics \\Problem Set 6\\}
\author{\\董建宇 ~2019511017}
\date{April 6}

\begin{document}
    
\maketitle
\newpage

\section{}
To determine the energy associated with this charge configuration, we need to determine the surface charge density on the surfaces of the two dielectrics. Assume the polarization surface charge density on the top dielectric is $\sigma_1$, the polarization surface charge density on the below dielectric is $\sigma_2$. We have that $$\sigma_1 = \vec{P_1}\cdot\vec{e}_z = \epsilon_0(\kappa_1-1)E_{1z},~\sigma_2 = \vec{P_2}\cdot(-\vec{e}_z) = -\epsilon_0(\kappa_2-1)E_{2z}$$\indent
We could calculate that $$\begin{aligned}
    E_{1z}&=-\frac{1}{4\pi\epsilon_0\kappa_1}\frac{qz}{\left(r^2+z^2\right)^{\frac{3}{2}}}-\frac{1}{4\pi\epsilon_0\kappa_1}\frac{qz}{\left(r^2+z^2\right)^{\frac{3}{2}}}\\
    E_{2z}&=-\frac{1}{4\pi\epsilon_0\kappa_2}\frac{qz}{\left(r^2+z^2\right)^{\frac{3}{2}}}-\frac{1}{4\pi\epsilon_0\kappa_2}\frac{qz}{\left(r^2+z^2\right)^{\frac{3}{2}}}
\end{aligned}$$\indent
Thus, we get that $$\begin{aligned}
    \sigma_1&=-\frac{\kappa_1-1}{\kappa_1}\frac{qz}{4\pi\left(r^2+z^2\right)^{\frac{3}{2}}}-\frac{\kappa_1-1}{\kappa_1}\frac{qz}{4\pi\left(r^2+z^2\right)^{\frac{3}{2}}}\\
    \sigma_2&=\frac{\kappa_2-1}{\kappa_2}\frac{qz}{4\pi\left(r^2+z^2\right)^{\frac{3}{2}}}+\frac{\kappa_2-1}{\kappa_2}\frac{qz}{4\pi\left(r^2+z^2\right)^{\frac{3}{2}}}
\end{aligned}$$\indent
So we could get that $$\sigma=\sigma_1+\sigma_2=\frac{\kappa_2-\kappa_1}{2\kappa_1\kappa_2}\frac{qz}{\pi\left(r^2+z^2\right)^{\frac{3}{2}}}$$\indent
We could determine the protential at point charge q is $$\varphi_q=\frac{1}{4\pi\epsilon_0\kappa_1}\frac{-q}{2z}$$\indent
The protential at point charge -q is $$\varphi_{-q}=\frac{1}{4\pi\epsilon_0\kappa_2}\frac{q}{2z}$$\indent
Thus, the energy associated with this charge configuration is $$E_{total}=\frac{1}{2}\left(q\varphi_q-q\varphi_{-q}\right)=-\frac{\kappa_1+\kappa_2}{16\pi\epsilon_0\kappa_1\kappa_2}\frac{q}{z}$$\indent
\subsection{$\kappa_1=\kappa_2=1$}
We have that if dielectrics 1 and 2 are replaced by vacuum, the total energy is $$E'=-\frac{1}{4\pi\epsilon_0}\frac{q}{2z}$$\indent
and this result makes sence.
\subsection{$\kappa_1=\kappa_2\neq 1$}
If $\kappa_1=\kappa_2=\kappa\neq 1$, we could get that the total energy is $$E''=-\frac{1}{4\pi\epsilon_0\kappa}\frac{q}{2z}$$\indent
and this result also makes sence.

\section{}
Since the system has axial symmetry, we choose spherical coordinates to describe the protential at all points in space and let the direction of dipole along the z-axis, so the protential has no connection with $\varphi$.\\\indent
According to the Laplace function, we have that$$\nabla^2V = \frac{\rho}{\epsilon}$$\indent
Assume the protential could be written as$$V(r,\theta) = S(r) \Theta (\theta)$$\indent
So we could get that the place in the dielectric sphere is $$\frac{1}{S}\frac{d}{dr}\left(r^2\frac{dS}{dr}\right) + \frac{1}{\Theta \sin\theta}\frac{d}{d\theta}\left(\sin\theta \frac{d\Theta}{d\theta}\right) = -\frac{q}{\epsilon_0\kappa}\left(\delta\left(\vec{r}-\frac{l}{2}\vec{k}\right)-\delta\left(\vec{r}+\frac{l}{2}\vec{k}\right)\right)$$\indent
The boundary conditions are V is continuous at r=R and the eletric field divided by $\epsilon$ is continuous at r=R. Which means $$V_{in}(R,\theta)=V_{out}(R,\theta),~\epsilon_0\kappa\left.\frac{\partial V_{in}}{\partial r}\right\rvert_{r=R}=\epsilon_0\left.\frac{\partial V_{out}}{\partial r}\right\rvert_{r=R}$$\indent
If there exists the dielectric sphere, we could determine the protential at $r<R$ is $$V_0'=\frac{1}{4\pi\epsilon_0\kappa}\frac{p}{r^2}\cos\theta$$\indent
When there is not the dielectric sphere, we could determine the protential at any point at (r,$\theta$) is $$V_0=\frac{1}{4\pi\epsilon_0}\frac{p}{r^2}\cos\theta$$\indent
According to the knowledge we have learnt, we could get the general solution is $$\begin{aligned}
    V_{in}(r,\theta)&=\frac{1}{4\pi\epsilon_0\kappa}\frac{p}{r^2}\cos\theta + \sum_{l=0}^{\infty} A_lr^lP_l(\cos\theta)\\
    V_{out}(r,\theta)&=\frac{1}{4\pi\epsilon_0}\frac{p}{r^2}\cos\theta + \sum_{l=0}^{\infty}\frac{B_l}{r^{l+1}}P_l(\cos\theta)
\end{aligned}$$\indent
Using the boundary conditions, we could get that $$\begin{aligned}
    \frac{1}{4\pi\epsilon_0\kappa}\frac{p}{R^2}\cos\theta + \sum_{l=0}^{\infty} A_lR^lP_l(\cos\theta)&=\frac{1}{4\pi\epsilon_0}\frac{p}{R^2}\cos\theta + \sum_{l=0}^{\infty}\frac{B_l}{R^{l+1}}P_l(\cos\theta)\\
    \frac{-2}{4\pi}\frac{p}{R^3}\cos\theta + \sum_{l=0}^{\infty} \epsilon_0\kappa lA_lR^lP_l(\cos\theta)&=\frac{-2}{4\pi}\frac{p}{R^3}\cos\theta - \sum_{l=0}^{\infty} \epsilon_0 (l+1)\frac{B_l}{R^{l+2}}P_l(\cos\theta)
\end{aligned}$$\indent
Thus, we could get that for $l\neq 1$, there must have $B_l=A_lR^{2l+1}$ but according to the second equation we get that $$B_l=-\frac{l\kappa}{l+1}A_lR^{2l+1},~for ~l=0,1,2,\dots$$ \indent
Which means for $l\neq 1$, $A_l=B_l=0$. When l=1, $P_1(\cos\theta)=\cos\theta$, we could get that $$\begin{aligned}
    A_1=\frac{2(k-1)}{k(k+2)}\frac{p}{4\pi\epsilon_0R^3},~
    B_1=-\frac{k-1}{k+2}\frac{p}{4\pi\epsilon_0}
\end{aligned}$$\indent
So the protential inside the sphere $(r\leqslant R)$ is $$V_{in}(r,\theta)=\frac{1}{4\pi\epsilon_0\kappa}\frac{p}{r^2}\cos\theta + \frac{2(k-1)}{k(k+2)}\frac{p}{4\pi\epsilon_0R^3}r\cos\theta=\frac{p\cos\theta}{4\pi\epsilon_0\kappa r^2}\left(1+2\frac{r^3}{R^3}\frac{\kappa-1}{\kappa+2}\right)$$\indent
The protential outside the sphere $(r\geqslant R)$ is $$V_{out}(r,\theta)=\frac{1}{4\pi\epsilon_0}\frac{p}{r^2}\cos\theta - \frac{k-1}{k+2}\frac{p}{4\pi\epsilon_0}\frac{1}{r^2}\cos\theta=\frac{3}{k+2}\frac{p}{4\pi\epsilon_0 r^2}\cos\theta$$

\section{}
\subsection{}
By the definition, we have that the capacitance of the partly-oil-filled capacitor is $$C=\kappa C_0$$\indent
$C_0$ is the capacitance when there is not oil and we have that $C_0=\frac{Q}{V}$ Q is the charge that the capacitor taking when there is not oil. Then we need to determine the charge Q. Using Gauss' Law and let $\lambda$ be the charge per unit length, we get that $$E\Delta h 2\pi r=\frac{\lambda \Delta h}{\epsilon_0}$$\indent
So we could get the eletric field when a<r<b is $$\vec{E}=\frac{\lambda}{2\pi\epsilon_0r}\frac{\vec{r}}{r}$$\indent
Then we could determine the protential difference is $$V=\int_a^b \vec{E}\cdot\,d\vec{r}=\frac{\lambda}{2\pi\epsilon_0}\ln\frac{b}{a}$$\indent
So we get that $$C_0=\frac{\lambda w}{V}=\frac{2\pi\epsilon_0 w}{\ln\frac{b}{a}}$$\indent
Then the capacitance of the partly-oil-filled capacitor is $$C=\frac{2\pi\epsilon_0\kappa w}{\ln\frac{b}{a}}$$\indent
\subsection{}
Using $(L-w+h_0-h)$ replace w in the expression of $C_0$($h_0$ is the raising height), we get the capacitance of the partly-air-filled capacitor is $$C'=\frac{2\pi\epsilon_0 (L-w+h_0-h)}{\ln\frac{b}{a}}$$\indent
The capacitance of the partly-oil-filled capacitor is $$C''=\frac{2\pi\epsilon_0\kappa (w-h_0+h)}{\ln\frac{b}{a}}$$\indent
Since the two capacitors are shunt capacitors, we could determine the total capacitance is $$C_t=C'+C''=\frac{2\pi\epsilon_0}{\ln\frac{b}{a}}\left[L+(\kappa-1)(w-h_0+h)\right]$$\indent
We have that $$\frac{1}{2}V^2\frac{d C_t}{dh}=\rho\pi(b^2-a^2)gh$$$\rho$ is the density of oil \\\indent
So the raising height is $$h=\frac{\epsilon_0(\kappa-1)V^2}{\rho(b^2-a^2)g\ln\frac{b}{a}}$$

\newpage
a
\end{document}