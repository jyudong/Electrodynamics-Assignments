\documentclass[10.5pt]{article}
\usepackage{amsmath,amssymb,amsthm}
\usepackage{listings}
\usepackage{graphicx}
\usepackage[shortlabels]{enumitem}
\usepackage{tikz}
\usepackage[margin=1in]{geometry}
\usepackage{fancyhdr}
\usepackage{epsfig} %% for loading postscript figures
\usepackage{amsmath}
\usepackage{float}
\usepackage{amssymb}
\usepackage{caption}
\usepackage{subfigure}
\usepackage{graphics}
\usepackage{titlesec}
\usepackage{mathrsfs}
\usepackage{amsfonts}
\usepackage{indentfirst}
\usepackage{fancybox}
\usepackage{tikz}
\usepackage{algorithm}
\usepackage{algcompatible}
\usepackage{xeCJK}
\usepackage{extarrows}
\setCJKmainfont{Kai}

\title{Electrodynamics \\Problem Set 4\\}
\author{\\董建宇 ~2019511017}
\date{March 15}

\begin{document}
    
\maketitle
\newpage

\section{}
\subsection{}
Since the system has axial symmetry, we choose spherical coordinates to describe the protential at all points in space and the protential has no connection with $\varphi$.\\\indent
According to the Laplace function, we have that$$\nabla^2V = 0$$\indent
Assume the protential could be written as$$V(r,\theta) = S(r) \Theta (\theta)$$\indent
So we could get that$$\frac{1}{S}\frac{d}{dr}\left(r^2\frac{dS}{dr}\right) + \frac{1}{\Theta \sin\theta}\frac{d}{d\theta}\left(\sin\theta \frac{d\Theta}{d\theta}\right) = 0$$\indent
So both of the items are constants.$$\frac{1}{S}\frac{d}{dr}\left(r^2\frac{dS}{dr}\right) = l(l+1),~~\frac{1}{\Theta \sin\theta}\frac{d}{d\theta}\left(\sin\theta \frac{d\Theta}{d\theta}\right) = -l(l+1)$$\indent
So we could calculate that$$S(r) = Ar^l+\frac{B}{r^{l+1}},~~\Theta(\theta) = P_l(\cos\theta)$$\indent
Then the protential at all points could be written as $$V_l(r,\theta) = \left(Ar^l+\frac{B}{r^{l+1}}\right)P_l(\cos\theta)$$\indent
While $r\leqslant R$, to guarantee that when $r \to 0$, $V_l(r,\theta) \to V_0$, we need that B=0, so $V_l(r,\theta) = A_lr^lP_l(\cos\theta)$\\\indent
While $r > R$, to guarantee that when $r\to \infty$, $V_l(r,\theta) \to 0$, we need that A=0, so $V_l(r,\theta) = \frac{B_l}{r^{l+1}}P_l(\cos\theta)$\\\indent
So the protential function is $$V(r,\theta) = \begin{cases}
    \sum_{l=0}^{\infty} A_l r^l P_l(\cos\theta), & ~for ~r\leqslant R,\\
    \sum_{l=0}^{\infty} \frac{B_l}{r^{l+1}} P_l(\cos\theta), & ~for ~r>R.
\end{cases}$$\indent
Since the function is continuous at r=R, so we have $B_l = A_l R^{2l+1}$. \\\indent
Since we have the charge distribution on the disk has the form $1/\sqrt{R^2-s^2}$, we could assume that the surface charge density is $$\sigma = \frac{k}{\sqrt{R^2-s^2}}$$.\indent
Then we have that $$V_0 = \int_0^R \frac{1}{4\pi\epsilon_0}\frac{2\pi s\sigma }{s} \,ds = \frac{k\pi}{4\epsilon_0}, ~~~k = \frac{4\epsilon_0V_0}{\pi}$$\indent
So the surface charge density is $$\sigma = \frac{4\epsilon_0V_0}{\pi}\frac{1}{\sqrt{R^2-s^2}}$$\indent
Then we could calculate the protential along z-axis $$V(r,0) = \int_0^R \frac{1}{4\pi\epsilon_0}\frac{2\pi s\sigma}{\sqrt{r^2+s^2}} \,ds= \frac{2V_0}{\pi} \int_0^R \frac{s}{\sqrt{R^2-s^2}\sqrt{r^2+s^2}} \,ds$$\indent
With Taylor Expansion, we get \begin{align*}
    V(r,0) =&\frac{2V_0}{\pi} \int_0^R \frac{1}{\sqrt{R^2-s^2}} \left[1 + \sum_{n=1}^{\infty} (-1)^n \frac{(2n-1)!!}{n! 2^n} \left(\frac{r^2}{s^2}\right)^n\right] \,ds\\
    =&V_0 + \sum_{n=1}^{\infty} (-1)^n \frac{(2n-1)!!}{n! 2^n}\frac{2V_0}{\pi} \int_0^R \frac{1}{\sqrt{R^2-s^2} s^{2n}} \,ds ~r^{2n}
\end{align*}\indent
So we could get that\Large $$A_l = \begin{cases}
    V_0, & ~if ~l=0,\\
    (-1)^{l/2} \frac{(l-1)!!}{(l/2)! 2^{l/2}}\frac{2V_0}{\pi} \int_0^R \frac{1}{s^{l} \sqrt{R^2-s^2}} \,ds, & ~if ~l ~is ~even,\\
    0, & ~if ~l ~is ~odd.
\end{cases}$$\normalsize\indent
Then we could also get that\Large $$B_l = \begin{cases}
    V_0R, & ~if ~l=0,\\
    (-1)^{l/2} \frac{(l-1)!!}{(l/2)! 2^{l/2}}\frac{2V_0}{\pi} R^{2l+1} \int_0^R \frac{1}{s^{l} \sqrt{R^2-s^2}} \,ds, & ~if ~l ~is ~even,\\
    0, & ~if ~l ~is ~odd.
\end{cases}$$\normalsize\indent
So the protential at all points in space is\Large $$V(r,\theta) = \begin{cases}
    V_0 + \sum_{k=1}^{\infty} \left[(-1)^{k} \frac{(2k-1)!!}{k! 2^{k}}\frac{2V_0}{\pi} \int_0^R \frac{1}{s^{l} \sqrt{R^2-s^2}} \,ds\right]r^{2k} P_l(\cos\theta),\\
    ~~~~~~~~~~~~~~~if ~r\leqslant R,\\
    \frac{R}{r}V_0 + \sum_{k=1}^{\infty} \left[(-1)^{k} \frac{(2k-1)!!}{k! 2^{l/2}}\frac{2V_0}{\pi} R^{4k+1} \int_0^R \frac{1}{s^{l} \sqrt{R^2-s^2}} \,ds\right] r^{-(l+1)} P_l(\cos\theta),\\
    ~~~~~~~~~~~~~~~if~r>R.
\end{cases}$$\normalsize
\subsection{}
The capacitance of the disk is $$C = \frac{Q}{V_0} = \frac{1}{V_0} \int_0^R 2\pi s\sigma \,ds = 8\epsilon_0 R$$

\section{}

\section{}
Firstly, we calculate the protential at all points in space. According to the Problem1, we have $$V(r,\theta) = \begin{cases}
    \sum_{l=0}^{\infty} A_l r^l P_l(\cos\theta), & ~for ~r\leqslant R,\\
    \sum_{l=0}^{\infty} \frac{B_l}{r^{l+1}} P_l(\cos\theta), & ~for ~r>R.
\end{cases}$$ where $B_l = A_lR^{2l+1}$ and we have the boundary at the e spherical shell $$\left.\frac{\partial V}{\partial r} \right\rvert_{R+}  - \left.\frac{\partial V}{\partial r}\right\rvert_{R-} = -\frac{\sigma (\theta)}{\varepsilon_0}$$\\\indent
So we get that $$\sum_{l=0}^{\infty} (2l+1)A_lR^{l-1}P_l(\cos\theta) = \frac{\sigma (\theta)}{\varepsilon_0}$$\\\indent
Then we have\begin{align*}
    A_l =&\frac{1}{2\varepsilon_0 R^{l-1}}\int_0^\pi \sigma_0(\theta)P_l(\cos\theta)\sin\theta \,d\theta\\
    =&\frac{\sigma}{2\varepsilon_0 R^{l-1}} \left(\int_0^1P_l(x) \,dx - \int_{-1}^0P_l(x) \,dx\right)
\end{align*}\indent
We know that $$\int_{-1}^0 P_l(x)\,dx = (-1)^l\int_0^1 P_l(x)\,dx$$\indent
So \large$$A_l = \begin{cases}
    0, & ~if ~l ~is ~even,\\
    \frac{\sigma }{\varepsilon_0 R^{l-1}}\int_0^1 P_l(x)\,dx, & ~if ~l ~is ~odd.
\end{cases}$$
$$B_l = \begin{cases}
    0, & ~if ~l ~is ~even,\\
    \frac{\sigma }{\varepsilon_0}R^{l+2}\int_0^1 P_l(x)\,dx, & ~if ~l ~is ~odd.
\end{cases}$$
\indent\normalsize
\subsection{}
So we could calculate the protential at $\vec{r} = (6R,0,8R)$ is $$V\left(10R,\arcsin\left(\frac{3}{5}\right)\right) = \frac{\sigma R^3}{2\varepsilon_0 r^2} \sum_{k=0}^{\infty} \left(-\frac{1}{2}\right)^k \left(\frac{R}{r}\right)^{2k} P_{2k+1}\left(\frac{4}{5}\right)$$
\subsection{}
While $\vec{r} = \vec{0}$, we could easily get that $$V(0) = 0$$

\end{document}